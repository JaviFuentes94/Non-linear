% $Revision: 1.5 $
\documentclass[12pt]{article}    % Specifies the document style.
\usepackage{amsmath}
\usepackage{times}
\usepackage{t1enc}
\usepackage{url}
\textwidth 16cm
\textheight 24cm
\oddsidemargin 0cm
\topmargin -1cm
\def\x{{\bf x}}

\begin{document}

\section*{Course 02457 Non-Linear Signal
  Processing, Exercise 6}

This exercise is based on C. M. Bishop: {\em Pattern Recognition
and Machine Learning}, Chapter~5. Print and comment on the figures
produced by the software as outlined below at the {\bf
Checkpoints}. You are going to use the `Neural Classification'
Matlab toolbox programmed at Cognitive Systems at DTU Informatics.

\subsection*{Pattern recognition with neural networks}

The neural network for classification is designed to approximate posterior class probabilities.
This is achieved using the
transformation from the neural network
outputs $\phi$ to the probabilities $y$ by the normalized exponential
or so-called {\em softmax} function:
\begin{equation}
  \label{eq:softmax}
  y_k = \frac{\exp(\phi_k)}{\sum^{c}_j \exp(\phi_j)} \ .
\end{equation}
This function has a built-in redundancy:
If there are $c$ classes and we use $c$ outputs we can calculate the
output of the $c$'th unit from the sum of the others.
A way to overcome this redundancy is via a {\em modified softmax}
which only has $c-1$ outputs from the neural network (before the
softmax). The $c$'th output is then calculated from the others:
\begin{eqnarray}
  \label{eq:modsoftmax}
  y_k &=& \frac{\exp(\phi_k)}{\sum^{c-1}_j \exp (\phi_j) + 1} \ ,
  \qquad k = 1,2, \ldots, c-1 \\
  y_c &=& 1- \sum^{c-1}_k y_k \ .
\end{eqnarray}


\subsection*{Pima indian data set}

This is a data set where the task is to classify a
population of women according to the risk of diabetes (binary classification).
There are 7 input variables, 200 training examples and 332 test
examples. 68 (34\%) in the training set and 109 (32.82\%) in the test
set have been diagnosed with diabetes.
In Brian Ripley's textbook {\em Pattern Recognition and Neural Networks} he
states that his best method obtains about 20\% misclassification on this data set so this
is what we can use for reference.
The input variables are:

\begin{enumerate}
\item Number of pregnancies
\item Plasma glucose concentration
\item Diastolic blood presure
\item Triceps skin fold thickness
% \item serum insulin
\item Body mass index (weight/height$^2$)
\item Diabetes pedigree function
\item Age
\end{enumerate}


The target output is 1 for examples having diabetes (``positive'' diagnosis), and 0 for healthy subjects (``negative'' diagnosis).
The variables are not directly fed into the neural network. Instead
the mean and standard deviation are calculated on the training set and
these values are then used to normalize both the training and
the test set, by first subtracting the mean and the dividing by the standard deviation. This procedure is called standardization, see Bishop page 567.

It is not possible to view all 7 dimension in one plot.
A way to plot the data is to select two variables and plot them against
each other.
Another way is to use principal component analysis and plot the data projected one principal component against another.

\subsubsection*{Checkpoint 6.1}

Use the program {\tt main6a.m} to optimize a neural network and plot
the decision boundary and conditional probabilities. The heavy blue line is
for locating the contour line corresponding to the equal probability of the two
classes.

Run the network with different weight decays and comment on the difference in the
decision surfaces.
Create a flow chart of {\tt main6a.m} and its functions.


\subsection*{Entropic cost function}

One could use the squared error function  to optimize
the neural network for classification:
\begin{equation}
  \label{eq:square}
  E_{\text{square}} = 1/2 \sum_n^N \sum_k^c (y_k^n - t_k^n)^2
\end{equation}
The square error cost function is linked to the `additive Normal noise' hypothesis.
However, as ${\bf y}$ is supposed to model a probability and ${\bf t} = (0,1,0,..,0)$
is a position coded target vector,  the Gaussian
error is inappropriate. Instead 
we can write the conditional likelihood function:
\begin{equation}
  \label{eq:multinom}
  p({\bf t}^n|{\bf x}^n) =
   \prod_k^c \left[{y_k^n}\right]^{t_k^n}
\end{equation}
This will give the so-called {\em cross-entropy} cost function appropriate for classification.
The negative log-likelihood of this is for an entire data set of
independent samples:
\begin{equation}
  \label{eq:multipleentropy}
  E_{\mbox{\tiny entropy}} = -\sum_n^N \sum_k^c
   t_k^n \ln y_k^n
\end{equation}
\iffalse
For a two-class problem with just a single output this can also be
written as:
\begin{equation}
  \label{eq:twoentropy}
  E_{\mbox{\tiny entropy}} = -\sum_n^N
  \left[ t_k^n \ln y_k^n + (1-t_k^n) \ln (1-y_k^n) \right]
\end{equation}
\fi


\subsubsection*{Checkpoint 6.2}

The program {\tt main6b.m} calculates the cost function value associated
with one example for a two-class problem.
It sweeps the output of the neural network $\phi_1$ from -10 to 10.

What was the target $t_1$? How much do the two cost functions penalize a
``wrong'' decision.


\subsection*{Posterior probabilities}

In the neural network we model the posterior class probabilities $p({\cal C}_k|\x) $ directly. However we can also start from the class conditional densities $p(\x|{\cal C}_k)$ and then use Bayes theorem to convert this into posterior class probabilities: 
\begin{equation}
  \label{eq:posterior}
  p({\cal C}_k|\x) = \frac{p(\x|{\cal C}_k)\ p({\cal C}_k)}{p(\x)} \ .
\end{equation}
This gives us a way for generating ground truth data to test the ability of the neural network to learn posterior class probabilities. 

\subsubsection*{Checkpoint 6.3}

Use the program {\tt main6c.m} to plot posterior probabilities.
The data is very simple with one dimensional $\x$ and two classes each Gaussian distributed with
$\sigma_{\cal A}^2=\sigma_{\cal B}^2=1$, $\mu_{\cal A} = -2$ and
$\mu_{\cal B}=2$, thus
$p(\x|{\cal C}_{\cal A})~\sim~{\cal N}(-2,1)$ and $p(\x|{\cal C}_{\cal B})~\sim~{\cal N}(2,1)$.
There are 100 samples in ${\cal A}$ and 12 samples in ${\cal B}$, initially,
 change this to larger values (e.g. 500 and 60, to keep the prior probabilities
 the same) and observe how the posterior approaches
 the true posterior.

\subsection*{Pruning}

The program {\tt main6d.m} will train and prune connections to optimize a neural network on
the Pima indian data-set.
The network is a multi-layer perceptron with softmax normalization.

We base pruning on the so-called `optimal brain damage' (OBD) approach. The saliency of each
weight is estimated using a second order Taylor expansion of the cost function around its
minimum. This estimates the saliency (cost of removing the weight) of weight number $q$ as
$S_{q} = \frac{1}{2}\frac{\partial^2 E}{\partial w_{q}^2}w_{q}^2$.

The pruning procedure trains the initial configuration which fully connected between
inputs and hidden layer nodes and between hidden layer and outputs nodes. The saliencies of all weigths
are computed and the least salient weight is pruned (put permanently to zero). The pruned network
is retrained and the procedure repeated until a minimal configuration is reached. During
pruning the test error is measured and after pruning the best network in test
is selected.



\subsubsection*{Checkpoint 6.4}
Derive the saliency expression as the approximate cost of removing a single weight while keeping
the remaining weights fixed at their `optimal' values.

The best net is selected as the net with the lowest cost function value
on the test set.

Note that not all inputs are used, is the number of pregnancies
important for the diagnosis?

The target output is 1 for positive diagnosis (diabetes), and 0 for negative diagnosis (healthy).

What does the neural network say about the relation between age and
diabetes (positive/negative correlation)?

\subsubsection*{Challenge I (not part of curriculum)}

Modify the set-up of Checkpoint 6.3 to have different variances for the two Gaussians: $\sigma_{\cal A}^2=2$ and $\sigma_{\cal B}^2=1$. Calculate and plot the posterior probability in this case (hint: quadratic discriminant) , redo the experiment with 100+12 and 500+60 examples and comment on how the neural network approximates the posterior.

\subsubsection*{Challenge II (not part of curriculum)}

Drop-out \url{http://arxiv.org/abs/1207.0580} is a new regularisation technique for neural networks. Modify the neural network code so that it performs drop-out regularization and apply it with standard settings (50\% drop-out) to the Pima Indian dataset. Do you see any differences in the decision boundaries to the ones found in Checkpoint 6.1 and what is the test performance compared to the other schemes we have considered. Can you explain the mechanism behind drop-out and why this in some cases will lead to better performance?


\vspace{2cm}
DTU, 1999, 2007, 2013
Finn \AA rup Nielsen, Lars Kai Hansen and Ole Winther


\end{document}


%%% Local Variables:
%%% mode: latex
%%% TeX-master: t
%%% End:
