\documentclass[times,12pt]{article}    % Specifies the document style.
\usepackage{amsmath}

\textwidth 16cm
\textheight 24cm
\oddsidemargin 0cm
\topmargin -1cm

\def\xb{{\bf x}}
\def\zb{{\bf z}}
\def\wb{{\bf w}}
\def\Ac{\mathcal{A}}
\def\Dc{\mathcal{D}}
\def\mub{\text{\boldmath $\mu$}}
\def\Sigb{\text{\boldmath $\Sigma$}}
\def\Sb{\text{\boldmath $S$}}
\def\L{\text{\boldmath $\Lambda$}}
\def\Ub{{\bf U}}
\def\vb{{\bf v}}
\def\ub{{\bf u}}
\def\db{{\bf d}}
\def\gb{{\bf g}}
\def\Hb{{\bf H}}
\def\Wb{{\bf W}}
\def\squeeze{\itemsep=0pt\parskip=0pt}

\begin{document}
\section*{{\it COURSE 02457}\\[5mm] Non-Linear Signal Processing: Exercise 7}

This exercise is based on C. M. Bishop: {\em Pattern Recognition
and Machine Learning} Chapter~9. Print and comment on the figures
produced by the software as outlined below at the {\bf
Checkpoints}.

\subsection*{Density Estimation}

We observe a stochastic multi-channel $d$-dimensional signal $\xb$
and our aim is to model the density $p(\xb) \sim p(\xb|\wb)$, where
the family $p(\xb|\wb)$ is a given parametric density. The
training set is ${\cal D} = \{\xb_1,\xb_2,\xb_3,...,\xb_N\}$ and the
likelihood function for $\wb$ is given by $p({\cal D}|\wb) = \prod_{n=1}^{N}
p(\xb_n|\wb) $. The cost function is minus the log likelihood:
\begin{eqnarray}
E(\wb) = \sum_{n=1}^{N} - \log  p(\xb_n|\wb). \nonumber
\end{eqnarray}
The {\it test error} of a density model can be estimated by
evaluating the cost function on a test set.


The mixture of Gaussians model family is defined as
\begin{eqnarray}
p(\xb|\wb) = \sum_{k=1}^{K} p(\xb|\wb_k)\pi_k \nonumber \ ,
\end{eqnarray}
where each component density is a normal distribution $\wb_k = \{
\mub_k,\Sigb_k\}$. Here we will invoke a family of ``isotropic''
Gaussians, i.e., Gaussians with covariance matrices that are
scaled unit matrices,
\begin{eqnarray}
p(\xb|\mub_k,\sigma^{2}_k) &=&
\frac{1}{(2\pi\sigma^{2}_{k})^{d/2}}\exp\left(-\frac{||\xb
-\mub_k||^2}{2\sigma^{2}_k}\right) \ . \nonumber
\end{eqnarray}
%
The Expectation-Maximization algorithm is a general scheme for
maximum likelihood estimation. For the mixture of Gaussians it leads to
the iterative procedure discussed in Bishop page 438-439. In the E-step, the means, variances and mixing
proportions are fixed and we 
update all responsibilities $\gamma_{nk}\equiv p(k|\xb_n)$, $n=1,\ldots,N$ and $k=1,\ldots,K$ in parallel
$$
\gamma_{nk} = p(k|\xb_n) =
\frac{p(\xb|\wb_k)\pi_k}{\sum_{k'=1}^{K} p(\xb|\wb_{k'})\pi_{k'}} \ .
$$
In the M-step, the responsibilities are fixed and we update the means, variances and mixing
proportions. We can write the update compactly by introducing $N_k$ the effective number of points assigned to component $k$:
\begin{eqnarray}
N_k & = & \sum_{n=1}^{N} \gamma_{nk} \nonumber\\[2mm] 
\mub^{\rm new}_k &= &\frac{1}{N_k}  \sum_{n=1}^{N} \gamma_{nk} \xb_n  \nonumber\\[2mm]
(\sigma^{\rm new}_k)^{2} &= &\frac{1}{d N_k} \sum_{n=1}^{N}
\gamma_{nk} ||\xb_n-\mub^{\rm new}_k||^2\nonumber\\[2mm]
\pi_k^{\rm new} &= &\frac{N_k}{N} \ . \nonumber
\end{eqnarray}
The rules for
updating $\mub_k$ and $\sigma^{2}_k$ are very similar to the well
known rules for computing mean and variance of a normal
distribution, but here weighted by the probability $\gamma_{nk}$
that a given data point $\xb_n$ belongs to mixture component $k$.

The {\it K-means} clustering algorithm is an important
simplification of these rules obtained by using each data point
only once, namely for updating the Gaussian component which it is
most associated with and by letting all the component variances be
equal (Bishop pages 424-430, 438-439). In the context of the
K-means algorithm a component is often referred to as a {\it
cluster}.

\subsubsection*{Checkpoint 7.1}

Use the program {\tt main7a.m} to perform K-means analysis on
synthetic two-dimensional data with three clusters. The program
creates two plots. The first shows the training points and the
current position of the cluster centers as time progress. You can
zoom to inspect the details of the convergence. The second figure
shows the assignment of points to clusters at the end of the
procedure. Is the final configuration sensitive to how the initial
cluster centers are located? You can change the distribution of
the initial clusters by changing the parameter {\tt initial-width}
and you can change the number of clusters {\tt K}. Sometimes you
will see that a cluster never moves away from its initial
position, why?


\subsubsection*{Checkpoint 7.2}

The program {\tt main7b.m} uses the Expectation-Maximization
algorithm to adapt a Gaussian mixture, as described above. The
program shows three plots. The first plot shows the training (blue)
and test (yellow) points, the second shows the temporal evolution
of the variance parameters and the training and test errors.
Create a flow chart of the program {\tt main7b.m} and its
functions. The test error is the mean negative log likelihood (the
cost function) estimated on the test set. Change the number of
clusters $K=2,3,4,5,6$ and inspect the temporal evolution and
final value of the test error. Explain how the Gaussian mixture
can over-fit.

\subsubsection*{Checkpoint 7.3}
In {\tt main7b.m} you can choose three different strategies for
initializing the cluster centers and initial variances.
 Set $K=5$ and run the program with the three different schemes. Describe the
strategies and their results. You will see a problem with the Gaussian mixture and
the EM algorithm for {\tt method 3}, where one component converge towards a very small variance
and is centered on a specific data point. Explain why this is a serious overfit.

\subsubsection*{Challenge I (not part of the curriculum)}

Modify the EM mixture of Gaussians program so that instead of the isotropic Gaussian full covariances $\Sigb_k$ is estimated. Visualise the one standard deviation contour of each of the components of the mixture. Hint: the points can created by $\Sigb_k^{\frac{1}{2}} {\bf u}(t)+\mub_k$, where ${\bf u}(t)=\left[ \begin{array}{c} \cos(t) \\ \sin(t) \end{array} \right]$ is taken as the points on the unit circle: $t \in [0,2\pi]$. The square root of the covariance  matrix can be found using the {\tt sqrtm}-function in Matlab. We will use these results in the Challenge to Exercise 8.

\subsubsection*{Challenge II (not part of the curriculum)}

Implement the K-mediods algorithm. Run it one the Gaussian simulated data using the Manhatten norm (sum of absolute difference for each dimension) as the cost funtion.

\subsubsection*{Challenge III (not part of the curriculum)}

Explain why K-means cannot be seen as a special limit of the mixture of Gaussians, that is there is not a specific setting of the covariance matrix where K-means and mixture of Gaussians give exactly the same updates in both the E- and M-steps.


\vspace{1cm} DTU, 1999 Lars Kai Hansen (2007 Ole Winther)


\end{document}
