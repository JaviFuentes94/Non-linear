
\documentclass[12pt]{article}    % Specifies the document style.
\usepackage{amsmath}
\usepackage{times}
\usepackage{t1enc}
\textwidth 15cm
\textheight 23cm
\oddsidemargin 0cm
%
\renewcommand{\floatpagefraction}{0.98}
\renewcommand{\textfraction}{0.02}
\renewcommand{\topfraction}{0.98}
\renewcommand{\bottomfraction}{0.98}
\def\xb{{\bf x}}
\def\zb{{\bf z}}
\def\wb{{\bf w}}
\def\Ac{\mathcal{A}}
\def\Dc{\mathcal{D}}
\def\mub{\text{\boldmath $\mu$}}
\def\Sigb{\text{\boldmath $\Sigma$}}
\def\Sb{\text{\boldmath $S$}}
\def\L{\text{\boldmath $\Lambda$}}
\def\Ub{{\bf U}}
\def\vb{{\bf v}}
\def\ub{{\bf u}}
\def\db{{\bf d}}
\def\gb{{\bf g}}
\def\Hb{{\bf H}}
\def\Wb{{\bf W}}
\def\ab{{\bf a}}
\def\Ab{{\bf A}}
\def\Bb{{\bf B}}

\begin{document}

\section*{Course 02457 Non-Linear Signal Processing, Exercise 9}

This exercise is based on C. M. Bishop: {\em Pattern Recognition
and Machine Learning}, section~2.5.

Print and comment on the figures produced by the software as outlined below at the {\bf
Checkpoints}.

\subsection*{Probability density function estimation using a kernel smoother}
A training set of N data points $D=\{\xb_1,\xb_2,..,\xb_N\}$ is
extrapolated `smoothened'  to test points $\xb$
\begin{eqnarray}
p(\xb|D,h) &=& \frac{1}{N} \sum_{n=1}^{N} k(\xb|\xb_n,h) \nonumber
\end{eqnarray}
with a `kernel' $k(\xb|\xb_n,h)$ given eg. by
\begin{eqnarray}
k(\xb|\xb_n,h) &=& \left(\frac{1}{2\pi h^2}\right)^{d/2} \exp
\left(-\frac{1}{2h^2}||\xb-\xb_n||^2\right)\nonumber
\end{eqnarray}
where the dimension of $\xb$ is $d$.
The parameter $h$ acts as a smoothing control. If
$h$ is small we roughly get a set of local `delta functions' centered on the training
data set, if $h$, on the other hand, is very large we get a near-uniform distribution.

\subsubsection*{Checkpoint 9.1}
We will use a validation set  - a test set for tuning of parameters - of $M$ samples to find $h$.
Explain why the function
\begin{eqnarray}
E(h)= \frac{1}{M}\sum_{m=1}^{M} -\log p(\xb_m|D,h) &=& \frac{1}{M}\sum_{m=1}^{M} -\log \frac{1}{N} \sum_{n=1}^{N} k(\xb_m|\xb_n,h) \nonumber
\end{eqnarray}
is a `test error'. Use the matlab script {\tt main9a.m} to generate data from a normal distribution
in $d=2$. What is the optimal $h$ for this data set.
Explain the structure of the densities obtained by the `optimal' $h$,
and $h$'s that are too small and too big.

How does the optimal $h$ depend on the training sample size $N$?

\subsection*{Signal detection using nearest neighbor methods}
We will use nearest neighbor methods for non-parametric classification.
Assume that a training set with $N$ class labeled samples is given.

In the K-nearest-neighbor (KNN) classifier we classify test points $\xb$ by voting among
the $K$ nearest neighbors in the training set. We implement this by brute force, i.e., simply by computing the
distance from the test point to all training points and sorting the distances.

A so-called `leave one out' estimate of the classification test error
can be obtained by computing the distances from every training point
to its $K$ neighbors (not including itself!)
and in turn estimate the classification error of the voting result
among the neighbors relative to the given training point's label.



\subsection*{Pima indian data set}

This is a data set where the task is to classify a
population of women according to the risk of diabetes (binary classification).
There are 7 input variables, 200 training examples and 332 test
examples. 68 (34\%) in the training set and 109 (32.82\%) in the test
set have been diagnosed with diabetes.
In Brian Ripley's textbook {\em Pattern Recognition and Neural Networks} he
states that his best method obtains about 20\% misclassifications on the test data set.
The input variables are:

\begin{enumerate}
\item Number of pregnancies
\item Plasma glucose concentration
\item Diastolic blood presure
\item Triceps skin fold thickness
\item Body mass index (weight/height$^2$)
\item Diabetes pedigree function
\item Age
\end{enumerate}

The target output is $1$ for examples diagnosed as diabetes, and $2$ for healthy subjects.

\subsubsection*{Checkpoint 9.2}
Explain how the `leave one out' error can be used for identifying the optimal number
of neighbors for voting. Use the matlab script {\tt main9b.m}
to classify the diabetes diagnosis data set.
What is the optimal $K$? How well is the KNN performance compared
to neural networks and other methods considered earlier in the course.

Consider classification from a subset of the seven input variable measures.
Estimate the performance for a few subsets, can you find a subset
with performance equal or better than that of the full feature set?

\subsection*{Local linear regression among nearest neighbors}

We can design a non-parametric function approximation scheme by
performing linear regression among the K-nearest neighbors of a given test point. We apply the method
to prediction of the sunspot test data set.

We use the linear model from exercise 3 and 4 to perform the estimation
in the test set.

\subsubsection*{Checkpoint 9.3}
Inspect the matlab script {\tt main9c.m}. Explain the role of the parameter `alpha'.
Why is it necessary to regularize the linear model?
What is the meaning of the parameter $d$ and what is optimal value of $d$.
Make a drawing that explains the algorithm conceptually, e.g.,  in a case with two-dimensional input.

Compare the  quality of the algorithm's predictions with
the neural network based predictions we found in exercise 5. What would happen if we used $K = N_{train}$?.

%\subsubsection*{Challenge (not part of the curriculum)}
%Inspect the local linear models in Checkpoint 9.3 for test points  with respect to mean, variance and 'outliers' (e.g., order test points according to how far away their local %linear models are from the mean). Where are the test points with 'outlier models', located in the sun-spot data?. Are the outliers consistent with respect to models trained with %different 'K' and 'alpha'?



\vspace{2cm}
\noindent DTU, November 2007, 2013, Lars Kai Hansen


\end{document}
