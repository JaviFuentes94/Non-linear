\documentstyle[12pt]{article}    % Specifies the document style.
\textwidth 16cm
\textheight 22cm
\oddsidemargin 0cm
\topmargin -1cm
\def\x{{\bf x}}
\def\w{{\bf w}}
\def\squeeze{\itemsep=0pt\parskip=0pt}

\begin{document}

\section*{{\it COURSE 02457}\\[5mm] Non-Linear Signal Processing: Exercise 1}

This exercise is based on C.M.\ Bishop: {\it Pattern Recognition
and Machine Learning}, Sections 1.2, 1.5. The objective of this
exercise is to become familiar with the relations between
probability densities and histograms, Bayes' theorem, conditional
distributions and decision rules.

Print and comment on the figures produced by the software {\sf main1.m, norm1d.m, probconfus.m}
as outlined below at the three {\bf Checkpoints}.


\subsection*{Densities and histograms}
A probability density function $p(x)$ specifies that the probability of the
variable $x$ lying in the interval between any two points $a,b$ is
\begin{equation}
P(x \in [a,b]) = \int_{a}^{b} p(x) dx
\end{equation}
If $\{x_1,x_2,...,x_N\}$ is a set of points, the histogram of this set,
 evaluated on the ordered set of points $[z_1,z_2,...,z_M]$ is defined
 \begin{equation}
H_j =  \sum_{x_{k}  \in [z_{j},z_{j+1}]} 1 \ \ ,j=1,...,M-1
\end{equation}
and the normalized histogram is given by
\begin{equation}
\tilde{H}_j = \frac{H_j}{\sum_{j'=1}^{M-1} H_{j'}}.
\end{equation}
The normalized histogram can be compared with the histogram approximation
to the density
\begin{equation}
P_j =  \int_{z_{j}}^{z_{j+1}} p(x) dx \ \ j=1,...,(M-1).
\end{equation}



We here focus on the density of the normal distribution:
\begin{equation}
p(x|\mu, \sigma^2) = \frac{1}{\sqrt{2\pi \sigma^2}}\exp \left( -\frac{(x-\mu)^2}{2\sigma^2}\right)
\end{equation}

\vspace*{3mm}

\noindent {\bf Checkpoint 1.1:} Use the program {\sf main1.m} to
illustrate the relation between densities, histogram
approximations to densities. Create samples from the univariate
normal density using {\sf randn.m} and compare the sample
histograms with the density of the normal distribution and with
its histogram.


\subsection*{Bayes' theorem}
For real variables $x$ and labels $C_k=1,...,c$, Bayes' theorem reads,
\begin{equation}
P(C_k | x) = \frac{p(x|C_k) P(C_k)}{p(x)}
\end{equation}

If the class-conditional distributions are univariate normals with individual parameter sets
we can use {\sf main1.m} to illustrate Bayes theorem.

\vspace*{3mm}

\noindent {\bf Checkpoint 1.2:} Define three univariate normals and plot the resulting densities. Set
the {\sf prior probabilities} $P(C_k)$ and plot the resulting $p(x)$ and
the posterior probabilities $P(C_k|x)$. Do this for different setting of the
prior probabilities and comment on the densities you get. Can you create a situation in which a class
has no points $x$ where it is most probable?


\subsection*{Decision boundaries}

A decision rule,  is a division of the
space of $x$ so that each point is uniquely associated with
a class $C_k$. We can set up a simple 1D decision rule by dividing the real axis into three
intervals $I_1 = ]-\infty, d_1], I_2 = ]d_1, d_2], I_3 = ]d_2, \infty[$.

One way to summarize the errors of a decision rule is the
 error confusion matrix $R_{j,k}$  defined, e.g., as
\begin{equation}
R_{j,k} = \int_{I_j} p(x|C_k) dx
\end{equation}

\vspace*{3mm}

\noindent {\bf Checkpoint 1.3:} Define the decision boundaries as above and compute
 the error confusion matrix. Do this for different decision boundaries, plot the
posteriors and the decision regions. Explain and comment on the confusion matrix.
The matlab program also computes an "alternative" confusion matrix $CC$, explain the role of this matrix.

%\subsection*{Challenge}
%
%Implement an error-reject mechanism as discussed by in Bishop section 1.10.1. Plot the relation
%between the error rate and the reject rate. Comment on the initial slope of the relation for small reject
%rates.


\vspace*{2cm}
\noindent DTU, September 2007,2013 \\[2mm]
Lars Kai Hansen


\end{document}
